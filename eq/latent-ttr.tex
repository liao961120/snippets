%' Equations interwoven with title text 
%' 
%' Used in slides to explain the potential of partitioning person ability
%' into a subjective component (e.g., report bias) and the true underlying 
%' ability (construct of interest).
\documentclass[12pt,preview,border=0]{standalone}
\usepackage[paperheight=10cm,paperwidth=14cm]{geometry}
\usepackage{amsmath}

% https://tug.org/FontCatalogue/mathfonts.html
\usepackage{heuristica}
\usepackage[heuristica,vvarbb,bigdelims]{newtxmath}
\usepackage[T1]{fontenc}
\renewcommand*\oldstylenums[1]{\textosf{#1}}

\newcommand\EV{V^\text{e}_{i}}
\newcommand\A{A_{i}}
\newcommand\N{N_{i}}
\newcommand\V{V_{i}}
\newcommand\Al{A_{\text{(0,1)}}}


\begin{document}

$$
\begin{aligned} 
    \N        & \sim \text{Gamma-Poisson}(\mu, \phi) \\
    \EV       &    = \text{EV}(\N) \\
    \lambda_i &    = \EV + \text{erf}(\A) ~ \EV ~ (1 - \frac{\EV}{\N}) \\
    \V        & \sim \text{Poisson}(\lambda_i)
\end{aligned}
$$
\\
$$
\begin{aligned}
    \mu_i     &    = \beta ~  \text{logistic}(\A) \\
    \N        & \sim \text{Gamma-Poisson}(\mu_i, \phi) \\
    \EV       &    = \text{EV}(\N) \\
    \lambda_i &    = \EV + \text{erf}(\A) ~ \EV ~ (1 - \frac{\EV}{\N}) \\
    \V        & \sim \text{Poisson}(\lambda_i)
\end{aligned}
$$

\end{document}
