\documentclass[preview,border=0]{standalone}
\usepackage{fontspec}
\usepackage{xeCJK}
\usepackage[paperheight=18cm,paperwidth=12.5cm]{geometry}
\usepackage{tikz}
\usetikzlibrary{arrows.meta, positioning,shapes.multipart}
\setCJKmainfont{微軟正黑體}

\begin{document}

% \begin{tikzpicture}[
%   level distance=1.5cm,
%   sibling distance=2.5cm,
%   every node/.style={font=\footnotesize},
%   edge from parent/.style={draw, -{Latex[length=3mm]}, thick, > = stealth}, % Arrow style
%   level 1/.style={sibling distance=3.5cm},
%   level 2/.style={sibling distance=3cm},
%   level 3/.style={sibling distance=2cm}
% ]

\begin{tikzpicture}[
    thick, % line style
    > = stealth, % arrow head style
    edge from parent/.style={draw, -{Latex[length=3mm]}},
    every text node part/.style={align=center},
    level distance=2.1cm,
    U/.style={circle, draw=white, inner sep=0pt, outer sep=2pt, minimum size=6mm},
    % level 2/.style ={level distance=3cm},
  level 1/.style={sibling distance=3.5cm},
  level 2/.style={sibling distance=1.3cm},
  level 3/.style={sibling distance=1.5cm,level distance=2.4cm,}
]

% Root node
\node{3至6歲孩子} [grow=0]
  child { node[U]{聽常組}
    child { node[U]{篩檢 \\ {\scriptsize(聽損未知)} }  
        edge from parent[->] 
        child { node[U]{\small $\textrm{P}(-|\textrm{聽常})$ \\ {\scriptsize 特異度} } edge from parent[->] }
    }
    edge from parent[->]
  }
  child { node[U]{聽損組}
    child { node[U]{篩檢 \\ {\scriptsize(聽損已知)}}  
        edge from parent[->] 
        child { node[U]{\small $\textrm{P}(+|\textrm{聽損})$  \\ {\scriptsize 敏感度}} edge from parent[->] }
    }
    edge from parent[->]
  };

\end{tikzpicture}

\end{document}
